\chapter{Projektplanung}
Das Kapitel behandelt die Projektplanung und beschreibt die erforderlichen Werkzeuge sowie die vorbereitenden Schritte, die für die Umsetzung des Projekts notwendig sind.

\section{Terminplanung}
\lipsum[2-4]

\section{Personalplanung}
\lipsum[2-4]

\begin{table}[h!]
\centering
\begin{tabular}{|l|l|l|l|}
\hline
\textbf{Team} & \textbf{Name} & \textbf{Tätigkeit} &  \textbf{Stunden}  \\ \hline
IDM        & name      & Projektumsetzung & 40       \\ \hline
IDM        & name    & Ansprechpartner, Projektübergabe & 8        \\ \hline
IDM        & name          & Projektdefinition, Projektunterstützung & 8        \\ \hline
IDM        & name                & Projektunterstützung                      & 8         \\ \hline
\end{tabular}
\caption{Personalplanung}
\label{tab:Personalplanung}
\end{table}

\section{Sachmittelplanung}
Die für das Projekt benötigten Materialien sind bereits verfügbar und müssen nicht extra angeschafft werden.
%nachfrage

\begin{longtable}{|l|p{10cm}|}
\hline
\textbf{Sachmittel}         & \textbf{Beschreibung}                                            \\ \hline
PC-Arbeitsplatz             & Umsetzung und Realisierung des Projektes sowie der Kommunikation \\ \hline
Microsoft Windows 10        & Betriebssystem für den Arbeitsplatz-PC                           \\ \hline
Microsoft Word              & Erstellen von Dokumenten und Anleitungen                         \\ \hline
Microsoft Teams             & Kommunikation zwischen Kollegen                                  \\ \hline
Microsoft Outlook           & Senden von Mails                                                 \\ \hline
Microsoft Azure/Entra       & Administrationsschnittstelle für Nutzerverwaltung                \\ \hline
Microsoft Edge              & gewählter Browser zur Bearbeitung der Konfigurationen            \\ \hline
Overleaf                    & Erstellung der Projektdokumentation in \LaTeX                    \\ \hline
\caption{Sachmittelplanung} \\
\end{longtable}

\section{Kostenplanung}
Die Kostenplanung erfolgt unter Berücksichtigung von Personal- und Lizenzkosten, wobei für die Berechnung der Personalkosten fiktive Stundensätze herangezogen werden, die sowohl die Löhne als auch die Lohnkosten sowie die Sachkosten umfassen.

\begin{table}[h!]
\centering
\begin{tabular}{|l|l|c|l|}
\hline
\textbf{Mitarbeiter} & \textbf{Stunden} & \textbf{Stundensatz} & \textbf{Gesamtkosten} \\ \hline
name   & 40     & 100€         & 4.000,00€   \\ \hline
name & 8      & 150€         & 1.200,00€   \\ \hline
name     & 8      & 150€         & 1.200,00€   \\ \hline
name            & 8      & 150€         & 1.200,00€   \\ \hline
Gesamt                 &        &              & 7.600,00€   \\ \hline
\end{tabular}
\caption{Personalkosten}
\label{tab:Personalkosten}
\end{table}


\section {Schutzbedarfsanalyse}
\lipsum[1]